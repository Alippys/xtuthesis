\chapter{参考文献著录规则及注意事项}

根据中华人民共和国国家标准《文后参考文献著录规则》(GB/T 7714—2005,中华人民共和国国家质量监督检验检疫总局、中国国家标准化管理委员会发布,2005年10月1日实施)[5]制定本规定。

\section{参考文献与脚注(注释)的区别}

参考文献按照提供目的划分,可分为引文文献、阅读型文献和推荐型文献。引文文献是著者在撰写或编辑论著的过程中,为正文中的直接引语(数据、公式、理论、观点等)或间接引语而提供的有关文献信息资源。阅读型文献是著者在撰写或编辑论著的过程中,曾经阅读过的文献信息资源。推荐型文献通常是专家或教师为特定读者、特定目的而提供的、可供读者阅读的文献信息资源[6]。学位论文中的参考文献主要指引文文献及阅读型文献,是论文的必要组成部分。

脚注与参考文献有所区别。参考文献是作者写作学术论文时所参考的文献,一般集中列于文末。参考文献序号用方括号标注,与正文中指示序号一致。脚注是对学术论文中某一特定内容所做的进一步解释或补充说明,一般排印在该页地脚,并用阿拉伯数字加圆圈标注[7]。

\section{参考文献著录方法[6]}

几种主要类型的参考文献(专著、专著中的析出文献、连续出版物、连续出版物中的析出文献、专利文献、电子文献等)的著录项目与格式要求如下:

\subsection{专著}

指以单行本或多卷册形式,在限定期限内出版的非连续出版物。包括以各种载体形式出版的普通图书、古籍、学位论文、技术报告、会议文集、汇编、多卷书、丛书等。其著录格式为:

\kaishu
[序号] 主要责任者. 题名: 其他题名信息 [文献类型标志(电子文献必备,其他文献任选, 以下同)]. 其他责任者(任选). 版本项. 出版地:出版者,出版年:引文起-止页码 [引用日期(联机文献必备,其他电子文献任选, 以下同)]. 获取和访问路径(联机文献必备, 以下同). ①

\songti
示例如下:

[1] 广西壮族自治区林业厅. 广西自然保护区. 北京: 中国林业出版社, 1993.

[2] 霍斯尼. 谷物科学与工艺学原理. 李庆龙, 译. 2版. 北京: 中国食品出版社, 1989:15-20.

[3] 王夫之. 宋论. 刻本. 金陵: 曾氏, 1865(清同治四年).

[4] 赵耀东. 新时代的工业工程师[M/OL]. 台北: 天下文化出版社. 1998 [1998-09-26]. http://www.ie.nthu.edu.tw/info/ie.newie.htm.

[5] 全国信息与文献工作标准化技术委员会出版物格式分委员会. GB/T 12450-2001 图书书名页. 北京: 中国标准出版社, 2002.

[6] 全国出版专业职业资格考试办公室. 全国出版专业职业资格考试辅导教材: 出版专业理论与实务•中级. 2004版. 上海: 上海辞书出版社, 2004: 299-307.

[7] World Health Organization. Factors regulating the immune response: report of WHO Scientific Group. Geneva: WHO, 1970.

[8] Peebles P Z, Jr. Probability, random variable, and random signal principles. 4th ed. New York: McGraw Hill, 2001.

[9] 郑开青. 通讯系统模拟及软件[硕士学位论文]. 北京: 清华大学无线电系, 1987.

\subsection{专著中的析出文献}

指从整本文献中析出的具有独立篇名的文献。其著录格式为:

\kaishu
[序号] 析出文献主要责任者. 析出文献题名[文献类型标志]. 析出其他责任者//专著主要责任者. 专著题名. 出版地:出版者,出版年:析出的起-止页码[引用日期]. 获取和访问路径.

\songti
示例如下:

[1] 白书农. 植物开花研究//李承森. 植物科学进展. 北京: 高等教育出版社, 1998:146-163.

[2] Weinstein L, Swertz M N. Pathogenic properties of invading micro- organism // Sodeman W A, Jr, Sodeman W A. Pathologic physiology: mechanisms of disease. Philadephia: Saunders, 1974:745-772.

[3] 韩吉人. 论职工教育的特点//中国职工教育研究会. 职工教育研究论文集. 北京: 人民教育出版社, 1985:90-99.

\subsection{连续出版物}

指一种载有卷期号或年月顺序号、计划无限期地连续出版发行的出版物,包括以各种载体形式出版的期刊、报纸等。其著录格式为:

\kaishu
[序号] 主要责任者. 题名:其他题名信息[文献类型标志]. 年,卷(期)- 年,卷(期). 出版地:出版者,出版年[引用日期]. 获取和访问路径(联机文献必备).

\songti
示例如下:

[1] 中国地址学会. 地质评论. 1936, 1(1)-. 北京: 地质出版社, 1936-.

[2] 中国图书馆学会. 图书馆学通讯. 1957(1)-1990(4). 北京: 北京图书馆, 1957-1990.

[3] American Association for the Advancement of Science. Science. 1883, 1(1)-. Washington, D.C.: American Association for the Advancement of Science, 1883-.

\subsection{期刊、报纸等连续出版物中的析出文献}

\kaishu[序号] 析出文献主要责任者. 析出文献题名[文献类型标志]. 连续出版物题名:其他题名信息,年,卷(期):页码[引用日期]. 获取和访问路径(联机文献必备).

\songti
示例如下:

[1] 张旭, 张通和, 易钟珍, 等. 采用磁过滤MEVVA源制备类金刚石膜的研究. 北京师范大学学报: 自然科学版, 2002, 38(4):478-481.

[2] 周桂莲, 许育彬, 杨智全, 等. 认清市场形势 化解“学报情结”: 我国农业学报的现状与发展趋势分析. 编辑学报, 2005, 17(3):209-211.

[3] 傅刚, 赵承, 李佳路. 大风沙过后的思考[N/OL]. 北京青年报, 2000- 04-12(14) [2002-03-06]. http://www.bjyouth.com.cn/Bqb/20000412/B/4216%5ED 0412B1401.htm.

\subsection{专利文献}

\kaishu[序号] 专利申请者或所有者. 专利题名:专利国别,专利号[文献类型标志(电子文献必备,其他文献任选)]. 公告日期或公开日期[引用日期(联机文献必备,其他电子文献任选)]. 获取和访问路径(联机文献必备).

\songti
示例如下:
[1] 刘加林. 多功能一次性压舌板: 中国, 92214985.2. 1993-04-14.

[2] 西安电子科技大学. 光折变自适应光外差探测方法: 中国, 01128777.2 [P/OL]. 2002-03-06 [2002-05-28]. http://211.152.9.47/sipoasp/zljs/hyjs-yx-new.asp? recid=01128777.2\&leixin=0.

\subsection{电子文献 }

以数字方式将图、文、声、像等信息存储在磁、光、电介质上,通过计算机、网络或相关设备使用的记录有知识内容或艺术内容的文献信息资源叫做电子文献,包括电子书刊、数据库、电子公告等。凡属电子图书、电子图书和电子报刊等中的析出文献的著录格式分别按上述有关规则处理,除此之外的电子文献的著录格式如下:

\kaishu[序号] 主要责任者. 题名:其他题名信息[文献类型标志/文献载体标志]. 出版地:出版者,出版年(更新或修改日期) [引用日期].获取和访问路径(联机文献必备).

\songti
示例如下:
[1] 萧钰. 出版业信息化迈入快车道[EB/OL]. (2001-12-19) [2002-04-15]. http://www.creader.com/news/200112190019.htm.

[2] Online Computer Library Center, Inc. History of OCLC[EB/OL]. [2000-01- 08]. http://www.oclc.org/about/history/default.htm.

[3] Scitor Corporation. Project scheduler[CP/DK]. Sunnyvale, Calif.: Scitor Corporation, c1983.

\section{参考文献表}

参考文献表用五号字,汉字用宋体,英文用Times New Roman体,行距采用固定值16磅,段前3磅,段后0磅。

参考文献的标注方式和参考文献表列法,可采用“顺序编码制”或“著者-出版年制”。确定采用某种方法后,在正文中的标注方法和列表中的写法是一一对应的。

参考文献表可以按“顺序编码制”组织,也可以按“著者-出版年制”组织。参考文献表按“顺序编码制”组织时,各篇文献要按正文部分标注的序号依次列出;参考文献表采用“著者-出版年制”组织时,各篇文献首先按文种集中,可分为中文、西文、日文、俄文、其他文种等5部分;然后按著者字顺和出版年排列。中文文献一般按汉语拼音字顺排列。

采用“顺序编码制”组织的参考文献表,每条文献的序号要加方括号“[  ]”;采用“著者-出版年制”组织参考文献时,每条文献不必加序号。采用悬挂格式,悬挂缩进2个汉字符。

\section{参考文献在正文中的标注法}

\subsection{顺序编码制}

1. 按正文中引用的文献出现的先后顺序用阿拉伯数字连续编码,并将序号置于方括号中,以上标形式放在句子的末尾。

2. 同一处引用多篇文献时,将各篇文献的序号在方括号中全部列出,各序号间用逗号,如遇连续序号,可标注起讫号“-”,例如:

张三[1]指出……,李四[2-3]认为……,形成了多种数学模型[7, 9, 11-13]……。

3. 同一文献在论著中被引用多次,只编1个序号,引文页码放在“[  ]”外,文献表中不再重复著录页码。例如:

张中国等[4]15-17……,张中国等[4]55认为……。

\subsection{著者-出版年制}

1. 正文引用的文献采用著者-出版年制时,各篇文献的标注内容由著者姓氏与出版年构成,并置于“(  )”内,放在正文中引用了该文献的句子末尾。倘若只标注著者姓氏无法识别该人名时,可标注著者姓名。集体著者著述的文献可标注机关团体名称。倘若正文语句中已提及著者姓名,则在其后的“(  )”内只须著录出版年。例如:

……(张中国, 2005),……张中国(2005)认为……。

2. 引用多著者文献时,对欧美著者只需标注第一个著者的姓,其后附“et al”;对中国著者应标注第一著者的姓名,其后附“等”字,姓名与“等”字之间留1个空格。例如:……(张中国 等, 2005)……。

3. 在参考文献表中著录同一著者在同一年出版的多篇文献时,出版年后应用小写字母a, b, c…区别。例如:

Kennedy W J, Garrison R E. 1975a. Morphology and genesis of nodular chalks and hardgrounds in the Upper Cretaceous of southern England. Sedimentology, 22:311-386.

Kennedy W J, Garrison R E. 1975b. Morphology and genesis of nodular phosphates in the Cenomaman of South-east England. Lathaia, 8:339-360.

4. 正文中多次引用同一著者的同一文献时,在正文中标注著者与出版年,并在“(  )”外以上标形式标注引文页码。例如:

……(张中国 等, 2005)15-17;……张中国 等(2005)55认为……。

\section{文献著录中应注意的若干问题}

\subsection{参考文献著录只有一个标准}

现行有效的关于参考文献著录的国家标准只有一个,即《文后参考文献著录规则》(GB/T 7714—2005),这是一个基础性的国家标准,适用于各个学科、各种类型的出版物。本指南就是根据该国家标准制订的。

今后我校研究生学位论文的参考文献著录和在正文中的标注方式,统一按照国家标准的规定,在“顺序编码制”和“著者-出版年制”两种方法中任选其一。

\subsection{正文中标注参考文献时的注意事项}

1. 用阿拉伯数字顺序编码的文献序号不能颠倒错乱;
2. 序号用方括号括起,同一处引用几篇文献,各篇文献的序号应置于一个方括号内,并用逗号分隔;

3. 多次引用同一作者的同一文献,只编1个首次引用时的序号,但需要将本次引用该文献的页码标注在顺序号的方括号外;

4. 文献表中的序号与正文中标注的文献顺序号要一一对应;

5. 作者可选择采用“顺序编码制”或“著者-出版年制”,但在同一篇论文中要统一。

\subsection{参考文献表著录时的注意事项}

1. 文后参考文献表原则上要求用文献本身的文字著录。著录西文文献时,大写字母的使用要符合文献本身文种的习惯用法;

2. 每条文献的著录信息源是被著录文献本身。专著、论文集、科技报告、学位论文、专利文献等可依据书名页、版本记录页、封面等主要信息源著录各个项目;专著或论文集中析出的篇章及报刊的文章,依据参考文献本身著录析出文献的信息,并依据主要信息源著录析出文献的出处;网络信息依据特定网址中的信息著录;

3. 采用顺序编码制组织的参考文献表,每条文献的序号要加方括号“[  ]”,采用悬挂格式,将序号悬挂在外;

4. 采用著者-出版年制组织参考文献时,每条文献不必加序号,采用悬挂格式,悬挂缩进2个汉字符;

5. 书刊名称不应加书名号,西文书刊名称也不必用斜体;

6. 西文刊名可参照ISO 4《信息与文献  出版物题名和标题缩写规则》的规定缩写,缩写点可省略。

\subsection{著录责任者的注意事项}

1. 责任者为3人以下时全部著录,3人以上可只著录前3人,后加“, 等” ,外文用“, et al” ,“et al”不必用斜体;

2. 责任者之间用“, ”分隔;

3. 欧美著者的名可缩写,并省略缩写点,姓可用全大写;如用中文译名,可以只著录其姓。例如:

Einstein  A 或 EINSTEIN  A(原题:Albert Einstein),韦杰(原题:伏尔特•韦杰);

4. 中国著者姓名的汉语拼音的拼写执行GB/T 16159—1996的规定,名字不能缩写。例如:Zheng  Guangmei 或 ZHENG Guangmei;

5. 不必著录主要责任者的责任。例如:

陈浩元. 科技书刊标准化18讲(原题:陈浩元主编. 科技书刊标准化18讲);

6. 不要求著录责任者的国别、所在朝代;

7. 机关团体名称应由上至下分级著录。例如:

中国科学院数学研究所(原题:中国科学院数学研究所);

Stanford University. Department of Civil Engineering(原题:Department of Civil Engineering , Stanford University)。

\subsection{参考文献表中数字的著录}

1. 卷期号、年月顺序号、页码、出版年、专利文献号等用阿拉伯数字。卷号不必用黑体。页码、专利文献号超过4位数时,不必采用三位分节法或加“,”分节,国外专利文献号中原有的分节号“,”在参考文献著录时删去;

2. 出版年或出版日期用全数字著录;如遇非公历纪年,则将其置于“( )”内。例如:2005-08-10,1938(民国二十七年);

3. 版本的著录采用阿拉伯数字、序号缩略形式或其他标志表示,第1版不著录,古籍的版本可著录“写本”、“抄本”、“刻本”等。例如:

3版(原题:第三版或第3版),5th ed.(原题:Fifth edition),2005版(原题:2005年版)。

\subsection{可作变通处理的著录项目}

1. 某一条参考文献的责任者不明时,此项可以省略(采用“著者-出版年制”时可用“佚名”或“Anon”);

2. 无出版地可著录[出版地不详]或[S.l.],无出版者可著录[出版者不详]或[s.n.];

3. 出版年无法确定时,可依次选用版权年、印刷年、估计的出版年,估计的出版年置于“[  ]”内;

4. 未正式出版的学位论文,出版项可按“保存地:保存单位,保存年”顺序著录。例如:

北京:中国科学院物理研究所, 2004

Berkeley:Univ of California Depart of Phys, 2005

5. 采用“著者-出版年制”标注时,联机文献的出版年根据更新或修改日期著录;倘若无更新或修改日期时,则可著录引用日期,并将其置于方括号内。例如:……(李中国, 2006)。……(张华, [2005])。Skinner G. 2001. A new code of ethics for librarians? ALAcodes and Johan Bekker’s proposals [EB/OL]. (2001-03-02) [2004-05-10]. http://www.redgraven....

6. 当文献中载有多个岀版地或多个岀版者时,只需著录第1个出版地或出版者;

7. 如果专著被作为阅读型和推荐型参考文献引用,其引文页码可以不著录;

8. 其他可以灵活处理的著录项目对GB/T 7714—2005 未作“必须”、“应当”等规定的著录项目,同一出版物可选定一种,并做到前后一致。
每条文献结尾可加“.”,图书的文献一般不加。

文献类型标志,非电子文献任选。

电子文献的引用日期非联机文献可以不著录。

西文的著者名、刊名一般采用缩写字母,但也可以采用全名;如果采用缩写字母,其缩写点可以省略,也可以保留。

其他责任者可不著录(如果著录,则须标注其责任,如“译”、“指导”等)。

纯电子文献的出版地、出版者、出版年可以省略(引用日期必须著录)。

期刊中析出文献的页码一般著录文章的起讫页,也可只著录起始页。

责任者的姓,其字母可以全大写,也可只首字母大写。GB/T 7714—2005并未做出全大写的规定,但有“大写字母的使用要符合文献本身文种的习惯用法”的指示,在外文文献中其实也是2种著录法并存。

\subsection{正确著录期刊文献的年份、卷、期、页}

\begin{itemize}
  \item 示例1:   年, 卷(期):页  2005, 10(2):15-20 
  \item 示例2:   年, 卷:页   2005, 35:123-129 
  \item 示例3:   年(期):页   2005(1):90-94   
  \item 示例4:   年(合期号):页  2005(10/11/12):15-20
  
  在同一刊物连载的文献,其后续部分不必另行著录,可在原参考文献后直接注明后续部分的年份、卷、期、页。
  
  \item 示例5:  年, 卷(期):页; 年, 卷(期):页
  
  2005, 15(1):12-15; 2005, 15(2):18-20
\end{itemize}

\section{文献类型、电子文献载体类型及其标志代码说明}

电子文献类型和载体类型标志是必备的著录项目。非电子文献类型可以省略。

学位论文类型的文献必须明确标注。中文硕士学位论文标注[硕士学位论文],中文博士学位论文标注[博士学位论文],外文学位论文标注[D]。

国标GB/T 7714—2005列出的文献类型标志如下:普通图书 M,会议录 C,汇编 G,报纸 N,期刊 J,学位论文 D,报告R,标准 S,专利 P,数据库 DB,计算机程序 CP,电子公告 EB。

会议录C指座谈会、研讨会、学术年会等会议的文集;汇编G指多著者或个人著者的论文集;S标志的文献还包括政策、法律、法规等文件。

电子文献载体类型标志如下:磁带 MT,磁盘 DK,光盘 CD,联机网络 OL。