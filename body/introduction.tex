\pagenumbering{arabic}
\chapter{引~~言}\label{chap:introduction}

\section{编写说明}
\cite{Arteaga-Falconi:ITIM:2016}\cite{Barni:ITIFS:2011}
研究生学位论文是作者攻读研究生期间研究成果的总结,是衡量作者是否达到研究生水平的重要依据。同时,研究生学位论文也是反映最高层次学历教育水平的学术作品,学校图书馆、中国学术期刊网等将作为学术资料长期保存,供同行学者、后续研究者查阅和参考。因此,要求学位论文文字正确,语言通顺,数据可靠,表述清晰,图、表、公式、单位等符合规范要求。同时,作为湘潭大学的研究生学位论文,在符合国家关于学位论文编写规范要求的基础上,应有统一的格式。

为此,研究生处依据中华人民共和国国家标准《科学技术报告、学位论文和学术论文的编写格式》(GB/T 7713—1987)、《文后参考文献著录规则》(GB/T 7714—2005),参照《清华大学博士学位论文写作指南》,编写了《湘潭大学研究生学位论文写作指南》,供研究生撰写学位论文时参考。

本《写作指南》只是一个指导性的文件。相关学科有特殊要求的,由学位点在本《写作指南》的基础上编写出相应学科的《写作指南》,报校学位办备案后执行。

\section{学位论文的基本要求}
国家标准《科学技术报告、学位论文和学术论文的编写格式》(GB7713—1987)中指出:学位论文是表明作者从事科学研究取得创造性的结果或有了新的见解,并以此为内容撰写而成、作为提出申请授予相应的学位时评审用的学术论文[1]。

研究生学位论文是研究生在导师指导下独立完成的、系统完整的学术研究工作的总结,论文应体现出研究生在所属学科领域做出的创造性学术成果,应能反映出研究生已经掌握了相应学位层次要求的基础理论和专门知识,并具备了独立从事学术研究工作的能力。

\section{撰写学位论文的语言及文字}
研究生学位论文要求用汉语书写,所用汉字须符合国家语言工作委员会、中华人民共和国新闻出版署联合发布的《现代汉语通用字表》[2]。专用名词、术语可采用国际通用的代号,量及其单位所使用的符号应符合国家标准《国际单位制及其应用》(GB3100—1993)[3]、《有关量、单位和符号的一般原则》(GB3101—1993)[4]的规定。图、表中的图题、坐标轴、图例、表头等描述性的词组或语句须使用汉语,专用名词术语、物理量及其单位可使用符合规范要求的符号。

外国人来华留学生可以用英文撰写学位论文,但须采用中文封面,且应有不少于6000字的详细中文摘要。

\section{主要内容}
本《写作指南》包括以下四方面内容:第一部分引言,阐述编写本《写作指南》的目的,以及按规范撰写论文的重要性;第二部分,湘潭大学研究生学位论文的基本格式要求,包括论文由哪些部分组成,排列顺序、装订方式、页面设置等具体格式要求;第三部分,参考文献著录规则及注意事项;第四部分,研究生学位论文中一些主要部分的写作方法和要求。此外,在附录中提供了部分模板样式(中文封面、英文封面及学位论文原创性声明与版权使用授权书),供研究生撰写论文时参考。 