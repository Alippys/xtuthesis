\chapter{研究生学位论文格式基本要求}

\section{学位论文的各组成及装订顺序}

研究生学位论文应包含如下部分,其装订顺序如下:
\begin{itemize}
\setlength\itemsep{0pt}
  \item 中文封面
  \item 英文封面
  \item 学位论文原创性声明与版权使用授权书
  \item 摘要
  \item Abstract
  \item 目录
  \item 主要符号对照表(不需要的可不列此部分)
  \item 第1章(或引言),第2章,……,结论
  \item 参考文献
  \item 致谢
  \item 附录(不需要的可不列此部分)
  \item 个人简历、在学期间发表的学术论文及研究成果
\end{itemize}

以上各项各自独立成为一部分,每部分从新的一页开始。

“中文封面”、“英文封面”、“学位论文原创性声明与版权使用授权书”三部分单面印刷,不加页码。从“中文摘要”开始各部分双面印刷,各部分之间不必留空白页。

从“中文摘要”开始至“目录”(或“主要符号对照表”)结束,页码用罗马数字“\rom{1}、\rom{2}、\rom{3}……”表示;从“第1章”(或“引言”)开始至论文结束,页码用阿拉伯数字“1、2、3……”表示。

页码置于页面下部居中,采用Times New Roman五号字体,数字两侧不加修饰线。

硕士学位论文封面用白色铜板纸,博士学位论文封面用黄色木纹纸。

\section{学位论文各部分的写作要求}

\subsection{中文封面}

中文封面包含四部分内容,分别为:论文题目、申请学位的学科门类、作者及导师信息、论文成文打印的日期、学位论文相关信息。

\textbf{1.论文题目}

论文题目\textbf{严格控制在25个汉字(符)以内}。字体采用\textbf{一号黑体字,居中书写},一行写不下时可分两行写,并采用\textbf{1.25倍行距}。

\textbf{2.作者及导师信息}

此部分包括:研究生、指导教师、学院、学科专业。各部分填写内容如下:

研究生:填写论文作者姓名。

指导教师:填写导师姓名,后衬导师职称“教授”、“研究员”等。一般情况下,只写一名指导教师。

学院:填写所属学院的全名,例如:哲学与历史文化学院。

学科专业:按一级学科培养的学科填写一级学科名称,其他填写二级学科专业名称,学科专业名称以国务院学位委员会批准的学科专业目录中的学科专业名称为准。

作者及导师信息部分使用\textbf{三号仿宋字}。

\textbf{3.论文成文打印的日期}

此部分填写论文成文打印的日期,用\textbf{三号宋体汉字},不用阿拉伯数字。

\textbf{4.学位论文相关信息}

学校代码:10530

分类号:《中国图书分类法》的分类号(可向校图书馆二楼文献检索室查询)。

密级:公开型论文不需注明,涉密论文需按学校有关规定填写《湘潭大学涉密学位论文审批表》进行审批。

学号:同等学力采用受理编号。

此部分填写内容使用小四号Times New Roman。

\subsection{英文封面}

英文封面的内容与中文封面相对应,题名使用Arial字体,字号20pt,加粗,居中书写。 其它信息内容使用\textbf{小四号Times New Roman}。

\subsection{学位论文原创性声明与版权使用授权书}

该部分内容可以直接下载本《写作指南》附件的Word文档,相应地复制到自己的论文中即可,在提交论文送审时研究生和导师都必须签署姓名。

\subsection{中、英文摘要}

中文摘要部分的标题为“摘要”,用黑体三号字,居中书写,单倍行距,段前空24磅,段后空18磅。摘要内容用小四号宋体字书写,两端对齐。

中文摘要控制在\textbf{800-1000}汉字(符),且篇幅限制在一页内书写。留学生用英文撰写学位论文时,中文摘要应不少于6000汉字(符),但英文摘要仍控制在500-800单词。

论文摘要中不要出现图片、图表、表格或其他插图材料。

论文的关键词,是为了文献标引工作从论文中选取出来用以表示全文主题内容信息的单词或术语,不超过5个,每个关键词之间用分号间隔。

英文摘要部分的标题为“Abstract”,用Arial体三号字,居中书写,单倍行距,段前空24磅,段后空18磅。摘要内容用小四号Times New Roman字体书写,两端对齐,标点符号用英文标点符号。“Key Words”与中文摘要部分的关键词对应,每个关键词之间用分号间隔。

论文摘要的中文版与英文版文字内容要对应。可双面打印。

\subsection{目录}

目录是论文主体内容各组成部分章、节序号和标题行按顺序的排列,列至二级节标题(例如2.2.5)即可。目录内容从第1章(或引言)开始,目录之前的内容及目录本身不列入目录内。目录中的章标题行采用黑体小四号字,固定行距20磅,段前空6磅,段后0磅;其他内容采用宋体小四号字,行距为固定值20磅,段前、段后均为0磅。

目录中的章标题行居左书写,一级节标题行缩进1个汉字符,二级节标题行缩进2个汉字符。

\subsubsection{主要符号对照表}

如果论文中使用了大量的物理量符号、标志、缩略词、专门计量单位、自定义名词和术语等,应编写“主要符号对照表”。如果上述符号和缩略词使用数量不多,可以不设专门的“主要符号对照表”,而在论文中出现时随即加以说明。 

\subsection{正文}

此部分是论文的主体,包括:第1章(或引言),第2章,……,结论。书写层次要清楚,内容应有逻辑性。

\textbf{1.标题}

标题要重点突出,简明扼要。格式如下:

\begin{itemize}
  \item \textbf{各章标题,例如:“第1章  引言”。}
  
  章序号采用阿拉伯数字,章序号与标题名之间空一个汉字符。采用黑体三号字,居中书写,单倍行距,段前空24磅,段后空18磅。论文的摘要,目录,主要符号对照表,参考文献,致谢,声明,附录,个人简历、在学期间发表的学术论文与研究成果等部分的标题与章标题属于同一等级,也使用上述格式;英文摘要部分的标题“Abstract”采用Arial体三号字加粗。
  
  \item \textbf{一级节标题,例如:“2.1  实验装置与实验方法”。}
  
  节标题序号与标题名之间空一个汉字符(下同)。采用黑体四号(14pt)字居左书写,行距为固定值20磅,段前空24磅,段后空6磅。
  
  \item \textbf{二级节标题,例如:“2.1.1  实验装置”。}
  
  采用黑体13pt字居左书写,行距为固定值20磅,段前空12磅,段后空6磅。 
  
  \item \textbf{三级节标题,例如:“2.1.2.1  归纳法”。}
  
  采用黑体小四号(12pt)字居左书写,行距为固定值20磅,段前空12磅,段后空6磅。\underline{一般情况下不使用三级节标题}。
  
\end{itemize}

\textbf{2.论文段落的文字部分}

采用小四号(12pt)字,汉字用宋体,英文用Times New Roman体,两端对齐书写,段落首行左缩进2个汉字符。行距为固定值20磅(段落中有数学表达式时,可根据表达需要设置该段的行距),段前空0磅,段后空0磅。

\textbf{3.脚注}

正文中某句话需要具体注释、且注释内容与正文内容关系不大时可以采用脚注方式。在正文中需要注释的句子结尾处用①②③……样式的数字编排序号,以“上标”字体标示在需要注释的句子末尾。在当页下部书写脚注内容。

脚注内容采用宋体小五号字,按两端对齐格式书写,单倍行距,段前段后均空0磅。脚注的序号按页编排,不同页的脚注序号不需要连续。脚注处序号“①,……,⑩”的字体是“正文”,不是“上标”,序号与脚注内容文字之间空半个汉字符,脚注的段落格式为:单倍行距,段前空0磅,段后空0磅,悬挂缩进1.5字符;字号为小五号字,汉字用宋体,外文用Times New Roman体。

\subsection{参考文献}

``参考文献''四个字的格式与章标题的格式相同。参考文献表的正文部分用五号字,汉字用宋体,英文用Times New Roman体,行距采用固定值16磅,段前空3磅,段后空0磅。

每一条文献的内容要尽量写在同一页内。遇有被迫分页的情况,可通过``留白''或微调本页行距的方式尽量将同一条文献内容放在一页。

关于参考文献的著录格式以及在正文中的标注方法详细见第3章。

\subsection{致谢}

用于评审、答辩、审议学位及提交学校存档的论文,致谢对象限于对完成学位论文在学术上有较重要帮助的团体和人士,致谢内容限200字以内,标题为“致谢”。 

\subsection{附录}

附录是与论文内容密切相关、但编入正文又影响整篇论文编排的条理和逻辑性的一些资料,例如某些重要的数据表格、计算程序、统计表等,是论文主体的补充内容,可根据需要设置。

附录的格式与正文相同,并依顺序用大写字母A,B,C,……编序号,如附录A,附录B,附录C,……。只有一个附录时也要编序号,即附录A。每个附录应有标题。附录序号与附录标题之间空一个汉字符。例如:“附录A  北京市2003年度工业经济统计数据”。

附录中的图、表、数学表达式、参考文献等另行编序号,与正文分开,一律用阿拉伯数字编码,但在数码前冠以附录的序号,例如“图A.1”,“表B.2”,“式(C-3)”等。 

\subsection{个人简历、在学期间发表的学术论文与研究成果}

个人简历包括出生年月日、获得学士、硕士学位的学校、时间等;

在学期间发表的学术论文分以下三种情况分别列出:

1.已经刊载的学术论文(本人是第一作者,或者导师为第一作者本人是第二作者)。按照参考文献格式书写,并在其后加括号,括号内注明该文检索类型,检索号,期刊的影响因子等。如果该论文未被检索,在括号内注明期刊级别或属于何种检索源刊,例如中文核心期刊、SCI源刊、EI源刊等。

2.尚未刊载,但已经接到正式录用函的学术论文(本人为第一作者,或者导师为第一作者本人是第二作者)。按照参考文献格式书写。在每一篇文献后加括号注明已被××××期刊录用,并注明期刊级别或属于何种检索源刊,例如中文核心期刊、SCI源刊、EI源刊等。

3.其他学术论文。可列出除上述两种情况以外的其他学术论文,但必须是已经刊载或者收到正式录用函的论文。
研究成果可以是在学期间参加的研究项目、获得专利或获奖情况等。获得专利请注明专利名称、作者、专利号;奖项请注明获奖名称、颁奖部门、获奖时间及个人在获奖者中的名次。

\subsection{量和单位}

要严格执行国家技术监督局1993年12月27日批准的、1994年7月1日开始实施的国家标准GB3100-3102—1993有关量和单位的规定。

单位名称的书写,可以采用国际通用符号,也可以用中文名称,但全文应统一,不得两种混用。

\subsection{有关图、表、表达式}

图、表和表达式按章编号,用两位阿拉伯数字分别编号,前一位数字为章的序号,后一数字为本章内图、表或表达式的顺序号。两数字间用半角横线“-”或小数点“.”连接。例如“图2-1”或“图2.1”,“表5-6”或“表5.6”,“式(1-2)”或“式(1.2)”等等。

\begin{itemize}
  \item \textbf{图}
  
  图要精选,要具有自明性,切忌与表及文字表述重复。
  
  图要清楚,但坐标比例不要过分放大,同一图上不同曲线的点要分别用不同形状的标识符标出。
  
  图中的术语、符号、单位等应与正文表述中所用一致。
  
  图序与图名,例如:“图2.1  发展中国家经济增长速度的比较(1960-2000)”。
  
  图2.1是图序,是“第2章第1个图”的序号,其余类推。图序与图名置于图的下方,采用宋体11pt字居中书写,段前空6磅,段后空12磅,行距为单倍行距,图序与图名文字之间空一个汉字符宽度。
  
  图中标注的文字采用9~10.5pt,以能够清晰阅读为标准。专用名字代号、单位可采用外文表示,坐标轴题名、词组、描述性的词语均须采用中文。
  
  如果一个图由两个或两个以上分图组成时,各分图分别以(a)、(b)、(c)……作为图序,并须有分图名。

  \item \textbf{表}
  
  表中参数应标明量和单位的符号。为使表格简洁易读,建议采用三线表(必要时可加辅助线),即表的上、下边线为单直线,线粗为1.5磅;第三条线为单直线,线粗为1磅。
  
  表单元格中的文字一般应居中书写(上下居中,左右居中),不宜左右居中书写的,可采取两端对齐的方式书写。表单元格中的文字采用11pt宋体字,单倍行距,段前空3磅,段后空3磅。
  
  表序与表名,例如:“表3.1  第四次全国经济普查数据(北京)”。
  
  表3.1是表序,是“第3章第1个表”的序号,其余类推。表序与表名置于表的上方,采用宋体11pt字居中书写,段前空12磅,段后空6磅,行距为单倍行距,表序与表名文字之间空一个汉字符。
  
  当表格较大,不能在一页内打印时,可以“续表”的形式另页打印,格式同前,只需在表序前加“续”字即可,例如“续表3.1  第四次全国经济普查数据(北京)”。
  
  若在表下方注明资料来源,则此部分用宋体五号字,单倍行距,段前空6磅,段后空12磅。需要续表时,资料来源注明在续表之下。

  \item \textbf{表达式}
  
  表达式主要是指数字表达式,例如数学表达式,也包括文字表达式。
  
  表达式采用与正文相同的字号居中书写,或另起一段空两个汉字符书写,一旦采用了上述两种格式中的一种,全文都要使用同一种格式。表达式应有序号,序号用括号括起来置于表达式右边行末,序号与表达式之间不加任何连线。
  
  表达式行的行距为单倍行距,段前空6磅,段后空6磅。当表达式不是独立成行书写时,有表达式的段落的行距为单倍行距,段前空3磅,段后空3磅。
  
  文中的表、图、表达式一律采用阿拉伯数字分章编号,例如:“表3.2”,“图2.5”,“式(3-1)”等。若图或表中有附注,采用英文小写字母顺序编号,附注写在图或表的下方。
\end{itemize}

\section{页面设置与页码}

\subsection{页面设置}

中、英文封面的页面设置已经在2.2.1和2.2.2中进行了介绍。

论文除中、英文封面、学位论文原创性声明与版权使用授权书三页采用单面印刷,从目录开始(包括目录)后面的部分均采用A4幅面白色纸张双面印刷。

\subsection{页码}

页码从第1章(引言)开始按阿拉伯数字(1,2,3,……)连续编排,之前的部分(摘要,Abstract,目录,主要符号对照表等)用大写罗马数字(\rom{1},\rom{2},\rom{3},……)单独编排。页码位于页面底端,采用五号Times New Roman体数字居中书写。页码数字两侧不要加“-”等修饰线。